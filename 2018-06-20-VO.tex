\section{2018-06-20 VO}

\subsection{Example: Navigation Support}
Was braucht ein Navigationsgerät eigentlich?
\begin{itemize}
\item Karten!
\item Routenplanung (Graph, Kostenfunktion, Optimierung)? 
\item Localization (GPS, braucht man, weil Menschen sich verfahren)
\item Context-aware (Wird Dunkel, also auf Nachtmodus schalten)
\end{itemize}
Lokalisierung kann zu Beginn lange dauern - letzten Abschaltpunkt als Ausgangslage.
Navigation ausschließlich wenn Ziel bekannt, oder auch wenn Suche nach Restaurant, Sehenswürdigkeiten.
\noindent\fbox{%
\parbox{\textwidth}{%
\textbf{Storytime}:  
Zypern, suche nach Kloster auf Berg - mit Navi nicht auffindbar. Besser Leute in der
Umgebung fragen. Eintritt für Frauen, nach Stundenlanger suche, verboten. Information 
zu Beginn der Routenplanung haben wäre Interessant. 
}%
}
\noindent\fbox{%
\parbox{\textwidth}{%
\textbf{Storytime}:  
Offline Karten - eigentlich eine gute Sache. Aber, physikalischer Einfluss (Hitze)
haben auf mobile Navis einen negativeren Einfluss als auf Eingebaute, die für solche
Situationen getestet wurden.
}%
}
\subsection{Mobile w/o Localization}
Älteste mobile Geräte: mobiles Radio. Laut Literatur der Sony Walkman.
Evolution hin zu Mobiltelefon und Apple iPod und Kombination derselben in ein Gerät.
Alle nicht wirklich eine Lokalisierung. Alte Mobiltelefone hatten Möglichkeit, war aber nicht für User vorgesehen (Triangulation). 
Odometrie (Positionsfeststellung via Reifenumdrehungen und Lenkereinschlag - kommt aus der Robotik). 
Gyro Sensor, Winkel zu Objekten...
\subsection{Mobile with Localization}
Smartphones, Orts-basiertes Filtern von Informationen, Orts-basiertes Tracking.
%\paragraph
%\\
%\noindent\fbox{%
%\parbox{\textwidth}{%
%\textbf{Storytime}:  
%}%
%}
%\\
