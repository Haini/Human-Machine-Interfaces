\section{2018-05-02}

\paragraph{Design for Life}

Kontext ist wichtig.
Zusammensammeln der Daten.
Sticky Notes zum Gruppieren, Zusammenfassung der Projektarbeit am Computer sorgt
für weniger Kollaboration. 
Deswegen recht praktisch wenn man physisch interagieren kann (für z.B. Brainstorming).
\textbf{Contextual Design Models}: Verschiedene Aspekte von Arbeit und Leben der User erstellen.
Teambasierte Workshops - grober, nicht funktionaler Entwurf oft ausreichend (Mockups, Powerpoint). \\
\textbf{Interaction design patterns}: Mit existierenden, abstrakten Methoden beginnen
Storyboarding: Geschichte erzählen, ist leichter zu verstehen. Mit Bildern skizzieren
für ein leichtes und schnelles Verständnis. \\
\textbf{User Environment Design}: (Basis) Struktur und Funktion des Systems festlegen.
Testing and iteration: Validieren, iterieren von Prototypen ist ein Schlüsselkonzept.\\

\subsection{Usage-Centered Design}
Wie wird die Software verwendet werden?
ßenarien für die Verwendung erstellen.
User Centered vs. Usage Centered Design kann nicht wirklich strikt getrennt werden. 
\\
\noindent\fbox{%
\parbox{\textwidth}{%
\textbf{Storytime}: Flug buchen über Internet, Geld abheben am Bankoment. Wer hat
denn die Kontrolle über den Ablauf? Eigentlich das Gerät. Kann aber auch einfach 
als eine Partizipation / Kooperation zwischen Mensch und Maschine mit gemeinsamen
Ziel betrachtet werden. 
}%
}
\\
\subsection{Essential Use Cases}
Wie oder wann kann man ein ßenario erstellen und beschreiben. 
Besser verständlich sind Geschichten, Mathematik und formale Abstraktion sind 
eher langsamer zu verstehen. 

\textbf{Essential Modeling}: Jetzt aber doch ein abstraktes Modell erstellen! 
Wie kommt man überhaupt zu einem neuen Entwurf? Was ist die Essenz der Verwendung?

Abstrakte Vision wie es werden soll, von dort weg konkretisieren. Hat aber eine
längere Vorlaufzeit - Ergebnisse sind erst später zu sehen. Schlecht für das
Management.


Nochmal das ATM Beispiel. War schon jemand beim Kassierer in der Bank Geld abheben? 
Prinzipiell funktioniert es eigentlich recht ähnlich. 
Eine abstrakte Darstellung funktioniert sowohl für den Kassierer als auch für den Bankomaten. 

Abstrakte Komponenten für den Inhalt des UIs. Abstrakte Komponenten des UIs,
Platzhalter für die tatsächliche Implementierung. 

Use Case ist die Zusammenfassung von mehreren Schlüsselverfahren.

\subsection{Process of Content Modeling}
Sprache und Jargon des Users sollte eigentlich nicht verwendet werden - andererseits ist dass dann eigentlich doch User-Centered.
Immer wenn die Maschine irgendetwas tun muss ist eine Funktion dafür notwendig.
Was ist aber notwendig wenn der Anwender dran ist. Funktionen des Users eigentlich
recht seltsam. Besser vielleicht: Was sind die Aufgaben des Users?
Beim Bankomat bedienen muss der User Aufgaben erfüllen.
\parbox{\textwidth}{%
\textbf{Storytime}: Schwedische Kronen abheben - keine Ahnung vom Wechselkurs.
Zu hohe Beträge abheben - immer Rückmeldung darauf, dass es zu viel ist - aber nicht
welchen Betrag man denn maximal abheben kann.

In Rom wird die Karte nach 30 Sekunden Inaktivität eingezogen.

Aufforderung dazu die Karte mitzunehmen, denn man ist ja wegen dem Bargeld da.
Dafür vergisst man dann vielleicht doch das Bargeld. 
}%
\\
Alles Aufgaben die der Mensch erfüllen muss. 

\paragraph{Essential Use Case - Video Store example}
Was passiert wenn wir 2 oder mehr Filme ausborgen wollen? Wird relativ mühsam
wenn man den gesamten Prozess für jeden Film durchlaufen muss. 

Lösung: Erstellen einer Content Navigation Map. Zeigt Interaktionen und Reihenfolgen
besser auf. Eventuell generieren von User Interfaces daraus.


\paragraph{Concrete vs. Abstract}
Persona: Konkreter Name für User (John) vs. User X. 

Zu viele Details können den Design Prozess stark einschränken. 
Abstrakte Beschreibungen sorgen für mehr Möglichkeiten.

Fokus von User vs Usage Centered ist ein anderer. Man kann für beides entwerfen.
