\section{2018-03-14 VO}

\paragraph{Coffee Machine}
Institutskaffeemaschine, eine gute Kaffeemaschine eigentlich. 
Sehr stabil und schafft den Durchsatz den das Institut braucht.
Kaffeebohnen sind leider endlich vorhanden - muessen nachgefuellt werden.
Meldung "Bohnenbeh. fuellen" kommt, darunter steht aber "Betriebsbereit".
Bohnen einfuellen, logisch. Aber was macht man dann?

\textit{Kurzer exkurs zu hochaufloesenden Bildschirm im Buero, geht nicht mehr - HDMI Kabel
umtauschen etc. Techniker findet auch nicht. Anstecken und Abstecken bringt 
auch nichts. Im Technikerzimmer ging er dann, ihn zurueck tragen hat dann auch noch 
immer funktioniert. Absteckdauer wohl zu kurz.}

Bedienungsanleitung fuer Kaffeemaschine suchen ist ein hoffnungsloser Auftrag.
Jemanden Fragen hilft vielleicht auch. Man muss aber 6 Sekunden auf den Knopf druecken!
Woher soll man das denn wissen??? 6 Sekunden sind aber sehr lang! Mit herumprobieren
findet sich eher keine Loesung. 

\paragraph{Web-pages for e-Shopping}
Keine Synonyme in der Suchmaschine eingebaut (Paradeiser, Tomate).
Suchergebnisse enthalten Dinge die nicht mehr auf Lager sind.
Zurueck gehen auf die Suchergebnisse laesst die Seite wieder ganz nach oben scrollen.
Wenn man ein Produkt in den Einkaufswagen legt, sind die Suchergebnisse weg.
Lieferkriterien werden begrenzt.

\paragraph{OEBB Redesign}
Wo findet man die Fahrplaene?\\
Wie vergleicht man Preise?\\

Wo nutzt man die Website? Am Telefon oder am Tablet oder am PC. Sieht ueberall
sehr gleich aus. Ist definitv fuer kleine Bildschirme geddacht. 

\textit{Kleiner Exkurs zum Bank Austria Konto fuer ein Projekt, SAP etc... Erfahrungen
mit dem Bank Austria User Interface? Noch schlimmer als OEBB! Groszer Bildschirm, 
keine gute Uebersicht, rauf runter scrollen und sehr schwierig zu navigieren. 
Problem ist noch immer nicht geloest mit verschieden groszen Bildschirmen. Stichwort
Reponsive Design, geht in die richtige Richtung - Problem der Widget Auswahl}

\paragraph{Defintion of Usability}
ISO Norm Definition, siehe Folien. Interpretation? Unterschied zwischen Effektiv
und Effizient: Kriterium erstellen, Kosten/Nutzen - Optimal.
Effektiv: Aufgabe wird erfuellt - Analogie: Bild haengt an der Wand, egal mit welchem Aufwand.
Effizient: Minimaler Aufwand - .

Menschen loesen Aufgaben eher nicht optimal - Stichwort Rubriks Cube.

Wirtschaftsnobelpreistraeger: satisficing (Herbert A. Simon).

Ziel muss auf jeden Fall erreicht werden! \textit{Specified Users} sind schwere
Definitionsgruppe - kann ja eigentlich jeder machen.
Definition ist sehr Vollstaendig, fast schon zu viel.

\textbf{Useful} - Nuetzlichkeit. Requirements erzeugen / verlangen ein useful System.
HMI setzt sich mehr mit \textbf{usable} auseinander.

Beispiel EKG fuer nuetzliche Geraete, die schwer zu bedienen sind. 
SAP ist auch nuetzlicher aber sehr kompliziert zu verwenden. Zuerst in der Industrie,
und jetzt auch auf der Universitaet! Verbindung von Uni Laptops zu Bunderechenzentrum
ist ebenfalls keine leichte Aufgabe. Und wenn man dann mal im SAP ist weisz man 
auch nicht so recht was man tun soll. Philosophie ist auf Buchhalter ausgelegt. 
Beispiel Geld ist Negativ wenn man Eines hat.
Beispiel Waschbecken mit Sensor fuer Wasserfluss - gute Idee, oft schwer zu aktivieren.

Was Usablity nicht ist vs. was Usability ist.

Guter Artikel auf Folie Nr. 18.

\paragraph{Usability Evaluation}
Um Usability Probleme zu finden, muss man Software auch testen! Analogie zu Bug in 
der Software. Sorgfaeltige Evaluation!

Fragen nach subjektivem Feedback. Frustrierte User vermeiden.

\paragraph{Euro Muenzen}
Findet man die Muenzen die man leicht? Gleiche Farbe, gleiches Layout, aehnliche
Groeszen. Bei Schilling war das irgendwie einfacher, waren Blind zu erfassen in der 
Geldboerse. Andere Farbgebungen, Groesze und Gewicht. Usability Problem!

Geldboerse wird sehr schnell sehr schwer, in den USA gibt es 1 Dollar Scheine.

Viele Usability Probleme sind keine mehr, wenn man sie kennt. \textit{Exkurs, Dekan 
berichtet ueber ESS System fuer Urlaube, Antragsgenehmigungen etc. Ist in SAP integriert,
wird bald entfernt. Hatte alle Probleme dieser Welt. Kein einziges Problem wurde geloest.
Dekan ist mittlerweile traurig, dass es abgeloest wird. System nicht beser, Mensch lernt
damit umzugehen. Bei Reisen: Tag, Uhrzeit Abflug / Ankunft - Nach Eintrag des Abflugtermins
sollte automatisch zumindest der Tag des Abflugs ausgewaehlt werden. Loesbar, aber laestig.}

\paragraph{Noch ein Beispiel - Banken}
IBAN - lang, schwer zu merken. Mit Leerzeichen deutlich besser lesbar!
Naechste Stunde, keine Stunde!

