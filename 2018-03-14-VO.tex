%\section{2018-03-14 Usability / User-Experience}

\subsubsection{Coffee Machine}
Institutskaffeemaschine, eine gute Kaffeemaschine eigentlich. 
Sehr stabil und schafft den Durchsatz den das Institut braucht.
Kaffeebohnen sind leider endlich vorhanden - müssen nachgefüllt werden.
Meldung "Bohnenbeh. füllen" kommt, darunter steht aber "Betriebsbereit".
Bohnen einfüllen, logisch. Aber was macht man dann?

\textit{Kurzer Exkurs zu hochauflösenden Bildschirm im Büro, geht nicht mehr - HDMI Kabel umtauschen etc. Techniker findet auch nicht. 
Anstecken und Abstecken bringt auch nichts. 
Im Techniker zimmer ging er dann, ihn zurück tragen hat dann auch noch immer funktioniert. 
Absteckdauer wohl zu kurz.}

Bedienungsanleitung für Kaffeemaschine suchen ist ein hoffnungsloser Auftrag.
Jemanden Fragen hilft vielleicht auch. Man muss aber 6 Sekunden auf den Knopf drücken!
Woher soll man das denn wissen??? 6 Sekunden sind aber sehr lang! Mit herumprobieren
findet sich eher keine Lösung. 

\subsubsection{Web-pages for e-Shopping}
Keine Synonyme in der Suchmaschine eingebaut (Paradeiser, Tomate).
Suchergebnisse enthalten Dinge die nicht mehr auf Lager sind.
Zurück gehen auf die Suchergebnisse lässt die Seite wieder ganz nach oben scrollen.
Wenn man ein Produkt in den Einkaufswagen legt, sind die Suchergebnisse weg.
Lieferkriterien werden begrenzt.

\subsubsection{öBB Redesign}
Wo findet man die Fahrpläne?
Wie vergleicht man Preise?
Wo nutzt man die Website? Am Telefon oder am Tablet oder am PC. Sieht überall
sehr gleich aus. Ist definitiv für kleine Bildschirme gedacht. 
\textit{Kleiner Exkurs zum Bank Austria Konto für ein Projekt, SAP etc... Erfahrungen mit dem Bank Austria User Interface? Noch schlimmer als öBB! Grosser Bildschirm, keine gute Übersicht, rauf runter scrollen und sehr schwierig zu navigieren. 
Problem ist noch immer nicht gelöst mit verschieden großen Bildschirmen. Stichwort Responsive Design (*4), geht in die richtige Richtung - Problem der Widget Auswahl}

\subsection{Definition of Usability}
ISO Norm Definition, siehe Folien. Interpretation? Unterschied zwischen Effektiv
und Effizient: Kriterium erstellen, Kosten/Nutzen - Optimal.
Effektiv: Aufgabe wird erfüllt - Analogie: Bild hängt an der Wand, egal mit welchem Aufwand.
Effizient: Minimaler Aufwand - .

Menschen lösen Aufgaben eher nicht optimal - Stichwort Rubics Cube.

Wirtschaftsnobelpreisträger: satisficing (*5) (Herbert A. Simon).

Ziel muss auf jeden Fall erreicht werden! \textit{Specified Users} sind schwere
Definitionsgruppe - kann ja eigentlich jeder machen.
Definition ist sehr Vollständig, fast schon zu viel.

\textbf{Useful} - Nützlichkeit. Requirements erzeugen / verlangen ein useful System.
HMI setzt sich mehr mit \textbf{usable} auseinander.

Beispiel EKG für nützliche Geräte, die schwer zu bedienen sind. 
SAP ist auch nützlicher aber sehr kompliziert zu verwenden. 
Zuerst in der Industrie, und jetzt auch auf der Universität! 
Verbindung von Uni Laptops zu Bundesrechenzentrum ist ebenfalls keine leichte Aufgabe. 
Und wenn man dann mal im SAP ist weiß man auch nicht so recht was man tun soll. 
Philosophie ist auf Buchhalter ausgelegt. 
Beispiel Geld ist Negativ wenn man Eines hat.
Beispiel Waschbecken mit Sensor für Wasserfluss - gute Idee, oft schwer zu aktivieren.

Was Usability nicht ist vs. was Usability ist.

Guter Artikel auf Folie Nr. 18.

\subsection{Usability Evaluation}
Um Usability Probleme zu finden, muss man Software auch testen! Analogie zu Bug in 
der Software. Sorgfältige Evaluation!

Fragen nach subjektivem Feedback. Frustrierte User vermeiden.

\paragraph{Euro Münzen}
Findet man die Münzen die man leicht? Gleiche Farbe, gleiches Layout, ähnliche
Größen. Bei Schilling war das irgendwie einfacher, waren Blind zu erfassen in der 
Geldbörse. Andere Farbgebungen, Größe und Gewicht. Usability Problem!

Geldbörse wird sehr schnell sehr schwer, in den USA gibt es 1 Dollar Scheine.

Viele Usability Probleme sind keine mehr, wenn man sie kennt. \textit{Exkurs, Dekan 
berichtet über ESS System ü Urlaube, Antragsgenehmigungen etc. Ist in SAP integriert,
wird bald entfernt. Hatte alle Probleme dieser Welt. Kein einziges Problem wurde gelöst.
Dekan ist mittlerweile traurig, dass es abgelöst wird. System nicht besser, Mensch lernt
damit umzugehen. Bei Reisen: Tag, Uhrzeit Abflug / Ankunft - Nach Eintrag des Abflugtermins
sollte automatisch zumindest der Tag des Abflugs ausgewählt werden. Lösbar, aber lästig.}

\paragraph{Noch ein Beispiel - Banken}
IBAN - lang, schwer zu merken. Mit Leerzeichen deutlich besser lesbar!
Nächste Stunde, keine Stunde!

