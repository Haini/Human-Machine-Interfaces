\section{2018-05-09}

\paragraph{Interaction Design}
Was ist Interaction Design überhaupt?

Definition aus Buch, ist eher Mangelhaft. CSCW: Computer Supported Cooperative Work.

\begin{center}
\begin{tabular}{ |c|c c| } 
 \hline
 - & local & remote \\ 
\hline
 sym &  &  \\ 
 arg &  &  \\ 
 \hline
\end{tabular}
\end{center}

Autoren des Buchs wissen vielleicht gar nicht was User Requirements sind?
Es sollten immer Alternativen untersucht und evaluiert werden. Normalerweise die
beste Lösung nicht gleich offensichtlich, meistens gibt es gute Alternativen die
durch Prototypen evaluiert werden sollten (außer es gibt schon jemanden der viel
Erfahrung mit Themen hat).
\\
Was ist es: Design der Interaktion zwischen Menschen und Maschine. Steht dann auch 
in Beziehung mit Requirements Engineering.
Geht auch Richtung "task analysis" (wie funktioniert der Ablauf bisher, Analyse des
Workflows). Gemeint ist aber bereits der Design Prozess. 
\\
\paragraph{Our own approach}

Domain Specific Language - DSL. Story Telling um das Zoo Beispiel zu erklären. 
Automatisches Generieren von User Interfaces, funktioniert schon gut für kleine
Bildschirme.
Reihenfolge ist recht wichtig, nicht nur Sequenz sondern mehr Information.
Discourse Model - könnte auch Dialogmodell genannt werden.
Communicative Acts - Speech Act - Wenn Sprache auch die Realität verändert.

"Können Sie mir den Schlüssel zurück bringen?" -- ist ein Request, keine Frage!
Unterscheidung ist wichitg. Zu Request gibt es Accept oder Reject.
UML Klassendiagramm zur Darstellung. Pfeile stehen für Generalisierung und
Spezialisierung. 

\paragraph{Adjacency Pair}
Diamant von UML ist eine Aggregation.

\paragraph{RST Relation - Joint}
Statemachines um einen Ablauf zu definieren. RST - Rethorical Structure Theory.
Nicht nur deklarativ sondern auch prozedural. 

Prozedurales Konstrukt: Frage nach Username und Passwort. "IfUntil" Konstrukt
sorgt dafür, dass das Passwort mehrmals falsch eingegeben werden kann. Quasi Schleife.

\paragraph{Conceptual Discourse Metamodel}
Strichlierte Linie stellt Assoziation dar, mehrfache Vererbungen?
\paragraph{Integration and Use of Ontologies}
Wisen worüber ein Dialog geführt wird. Modell an Grenze zwischen Mensch und Maschine.

Interface zu Aktionen der Software, z.B durch Action Notification Model.
Applikationsspezifische Aktionen.

Prototyping per Knopfdruck ist eine große Stärke!

%\paragraph
%\\
%\noindent\fbox{%
%\parbox{\textwidth}{%
%\textbf{Storytime}:  
%}%
%}
%\\

