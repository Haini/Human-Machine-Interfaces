\section{2018-06-06 VO}

\paragraph{Gestures - On-screen gestures}

Stift und Finger basierte Gesten wie zum Beispiel Tippen, Ziehen, Wischen. 
Genau genommen sind solche Sachen keine Gesten - in der Technik bezieht man sich 
aber meistens auf "Remote Gestures" wenn man echte Gesten mit Händen und Armen meint.

\paragraph{Sounds}
Wird der Ton überhaupt verwendet? Navigationsgeräte mit Warnton bei
Geschwindigkeitsübertretungen - macht prinzipiell Sinn wenn man die Toleranz einstellen
kann.

Bei Fehlern ist es eine gemischte Sache, prinzipiell ist Feedback bei Fehlern eine gute
Sache, die Frage ist immer in welcher Form und Häufigkeit etwas dargestellt. 

Bei Autos, wenn nur noch Reservetank verwendet wird - Warnton, Warnblinkanlage?
Bei Scheibenwaschmittelstands Warnungen ist ein sehr lauter Ton vielleicht nicht
angemessen.

Angemessene Lautstärke / Intensität, schwierig einzuteilen, da es eine subjektive
Wahrnehmung ist.

\paragraph{Speech Output and Input}

\textbf{Sprachausgabe} ist sogenannter \textit{Canned Output} - Teile werden herausgenommen
und entsprechend zusammengesetzt. 

\textbf{Spracheingabe} - Training auf eine bestimmte Person früher wichtiger, heute nicht mehr
ganz so relevant. Probleme mit Umgebungsgeräuschen beheben - Mikrofon am Mensch oder 
sehr gute Filter / Mikrofone am Gerät.

\paragraph{Haptic Output}

Bringt große Vorteile wenn visuelle Möglichkeiten eingeschränkt sind - Abzählen von 
Tasten am Handy, finden vom richtigen Knopf...

\paragraph{Combined Modalities - CARE Properties}

\textit{\textbf{C}omplementarity} kombiniert verschiedene Modalitäten die in einem sich ergänzenden Weg
verwendet werden sollen.\\
\textit{\textbf{A}ssignment} zu einer bestimmten Modalität, ohne Wahl einer anderen
Modalität (QR Code Reader beim Einkaufen am Einkaufswagen, Produkt ist nach Scan gekauft,
egal ob man es dann auch wirklich in den Einkaufswagen legt oder nicht).\\
\textit{\textbf{R}edundncy}: Verschiedene Modalitäten mit der selben Ausdruckskraft werden 
verwendet (Zum Beispiel Blinken und Ton, wenn man an Produkt das man Kaufen will vorbei
geht).\\ 
\textit{\textbf{E}...}\\

Resultiert in Design Problem, welche Modalität verwende ich um welche Anfragen / Aufgaben 
zu kommunizieren? Viele verschiedene kombinationen Möglich.

\paragraph{Combined Modalities - Fusion for Multimodal UIs}

High-level fusion: Sehen von Autos in der Nacht, noch immer schwierig mit Sensoren die 
auf Tageslicht designed sind. Zusätzliche Integration von Radar, Lidar, Ultraschall um 
eine Verbesserung der Erkennungsrate zu erzielen. Kosten und praktische Integrierbarkeit
(Energiekosten, Platz) sorgen für Limitierungen.\\
Vertrauen in welchen Sensor, wenn ein Konflikt entsteht? \\
Erstellen einer Statemachine um verschiedene Eingaben zu parsen und zu kombinieren
um eine gemeinsame Repräsentation zu bauen. \\
Dabei gibt es zeitliche Randbedinungen etc.\\

\paragraph{Fission for Multimodal UIs}

Zerlegen von Ausgaben.

Interessant: Zusammenspiel zwischen den verschiedenen Sensoren ist sehr interessant,
kontextbasiertes Handeln ist gerade auch hier wichtig.

\paragraph{Designing UIs for Mobile Devices}

Localization in UIs - relativ neu und nützlich. Schon seit fast immer wusste 
zumindest der Mobilfunkbetreiber wo das Handy ist. Für den Nutzer hat sich daraus
kein zusätzlicher Nutzen ergeben.

\paragraph
\\
\noindent\fbox{%
\parbox{\textwidth}{%
\textbf{Storytime}: Verwendung von Navigationsgeräten - eingebaute Navigationsgeräte in 
Fahrzeugen sind leider nicht besonders beliebt. Google macht hier einen guten Job und 
ist möglicherweise schon recht nah am Monopol. 

Ist es aber grundsätzlich gut ein Navigationsgerät fix im Fahrzeug eingebaut zu haben?\\
\textbf{Nein}, weil Elektronik schnell altert und getauscht wird, zumindest schneller als
ein Auto.\\
\textbf{Ja}, weil die Integration ins Auto viel besser ist - Abgreifen von Lenkradeinschlag,
Geschwindigkeit etc. sorgt für eine bessere Überbrückung von Funklücken (Aka Odometrie).\\

Anzeige von mehr Informationen auf einem größeren Display ist ein großer Vorteil gegenüber 
kleineren Handy Bildschirmen. 

Nächster Schritt ist die Integration von Smartphones in das Auto (Carplay, Google Auto).
Extra gekaufte Navis werden eher sterben, Telefon und Auto Navi sind überlegen.
}%
}
\\

NÄchste Woche, 2018-06-12 keine VO!!


%\paragraph
%\\
%\noindent\fbox{%
%\parbox{\textwidth}{%
%\textbf{Storytime}:  
%}%
%}
%\\
