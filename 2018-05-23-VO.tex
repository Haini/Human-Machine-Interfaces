\section{2018-05-23}

\paragraph{User Study Design}

Nochmal Wiederholung der verschiedenen Flugbuchungs User Interfaces.

Nullhypothese bzgl. verschiedenen Screen Sizes wurde aufgestellt. Horizontal hat schlecht
abgeschnitten, aufgrund der schlechteren Entdeckbarkeit. Intuitiv scrollt man lieber nach
unten als zur Seite. 

Task Times für ein Tablet sind im Schnitt immer kleiner als auf einem iPod Touch. Naiver
Schluss: Grösserer Bildschirm erleichtert die Interaktion immer. Auch die Fehlerrate
sinkt. 

Am Telefon war das vertikale UI beliebt und am Tablet das horizontale UI. Zeigt auch dass
GUIs immer an die Größe des Geräts angepasst werden muss. Effizienzverlust durch
Verwendung von UIs die auf kleine Bildschirme optimiert sind - wenn sie auf großen
Bildschirmen verwendet werden. 

Verallgemeinerung vielleicht schwieirig - Problem Space an möglichen Versuchsaufbauten
sehr groß.

\paragraph{Multimodal Interfaces}
Modality: Physikalischer Kanal über den Mensch und Maschine interagieren / sich
gegenseitig wahrnehmen. 
Definiert sich über das physische Medium und den Weg seiner Darstellung.

Bei Telefon zum Beispiel das Vibrieren des Telefons - haptisches Feedback. Altes
Tastenhandy konnte fast blind navigiert werden, Tasten bieten gutes haptisches Feedback. 

Übliche Modalitäten sind zum Beispiel Bildschirm und Mäuse / Tastaturen. 

Gesten gäbe es auch - werden aber wenig verwendet? Warum? Sinnhaftigkeit von Gesten
bewegen sich mehr im Räumlichen - z.B für Roboter, "Komm her" Geste. 

Ton / Sound gibt es auch. Wann wird Ton am Laptop verwendet? Zu Hause, beim Arbeiten -
viele Töne - außer Fehlermeldungen - macht er eigentlich nicht. 

Spracheingabe und Ausgabe. Mittlerweile sehr nützlich. Google Duplex ist z.B. ganz nett,
lässt sich aber hauptsächlich über Scripting etc. lösen.

Haptische Ausgabe und viele mehr sind auch noch möglich.

\paragraph{Input Devices}
Haufenweise Möglichkeiten, Stimmbänder, Finger, Füße, ...

Verschiedene Optionen der Gestaltung und ihre Vor- und Nachteile (z.B Menu Selection, Form
Fill-in, Natürliche Sprache, Direkte Manipulation).

Bankomaten haben z.B. einen Touchscreen und die Knöpfe auf der Seite. Hauptinteraktion am
Touchscreen, Bestätigung dann erst wieder über haptische Taste. Inkonsistenz ist nicht
optimal. 

\paragraph{Presentation of Information}

Wie soll Information überhaupt präsentiert werden.

Auto Tachometer - Analog / Digital?

Uhr - Analog / Digital?

Leichte Abschätzbarkeit im Analogen, aber geringere Präzision. Ist ein Tradeoff. 

%\paragraph
%\\
%\noindent\fbox{%
%\parbox{\textwidth}{%
%\textbf{Storytime}:  
%}%
%}
%\\

