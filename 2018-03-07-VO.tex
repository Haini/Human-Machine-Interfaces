\paragraph{Einführung}
Es geht um Verständnis und nicht um Auswendig lernen
Programmieren kann man nicht durch Vorträge lernen, sondern nur durch Übung.
Diese VO ist allerdings ohne Übungsteil - entsprechend ist das Wissen / Verständnis
besonders wichtig. 
Diskussionen sind sehr erwünscht.

\paragraph{Folie 4}
Folien sind im TISS erhähltlich, sobald man sich angemeldet hat. Anmeldung wird
empfohlen.

Nicht jede Einheit wird stattfinden, entsprechende Updates gibt es allerdings
im TISS Kalender. 

Skriptum gibt es leider nicht, VO besteht aus diversen Büchern, Liste wird noch
hochgeladen, dienen eher als Referenz - muss man also nicht kaufen.

\paragraph{Folie 5}
Agenda und Plan. Steht prinzipiell auch im TISS.
Usability / Experience - Gefühl bekommen für Zusammenhang zwischen Computern und 
Menschen, sprich den Human Factor in der Technik verstehen.
Beispiele aus dem alltäglichen Leben von Prof. Kaindl, eigene Beispiele und Erfahrungen
einbringen.

Human Factors
Warum funktioniert der Mensch nicht wie eine Maschine? Entwurf kümmert sich also
nicht nur um Software Entwurf sondern auch um den Entwurf einer vernünftigen 
Interaktionsmöglichkeit mit dem Menschen. 

User-Centered ist Ansatz wie man so etwas konstruktiv erzielen kann. 

Interaction Design, auch Kaindl kann nicht genau sagen was das ist, verschiedene
Definitionen - Diskussion zu späterem Zeitpunkt, wenn Basis vorhanden.

Usability Test - alles muss getestet werden, auch User Interfaces. Wissenschaftliche
Vergleiche, Analyse von Problempunkten - in einem weiterführenden, optionalen Seminar
wird so etwas wirklich untersucht.

Multimodal Interfaces - 

Design UIs for Mobile Devices - Aktuelles und wichtiges Thema, Mobilität wichtig.

Human-Robot Interaction - Roboter kann groß sein, Gefährdung für den Menschen,
verschiedenste Aktionsmöglichkeiten (Aktuatoren, Sensorik).

Es folgen jetzt einige Beispiele.

\paragraph{Folie 3}
Control Interface wird gezeigt, ist für den normalen User nicht verständlich,
sehr individuell, von Maschinenbauern entworfen. Das Ziel ist es solche umständlichen
Interfaces durch Laptops / Handys zu ersetzen. Aber ist das gut? Kritisch gegenüber 
User Interface Klassen sein, nicht alles gutieren.

\paragraph{Folie 4}
Wir sollen in dem Studium die Grundlagen von User Interfaces lernen. Ist Kaindl wichtig.
Alte Leute sind benachteiligt, Beispiel Heizung: Soll Heizen wenn Kalt, Aus sein wenn warm.
Alte Heizung einfach einen Knopf für Sommer / Wintermodus gehabt. Neues Gerät angeschafft,
wollen Sommermodus einschalten - Warmwasser soll natürlich dann noch gehen.
Herr Kaindls Großeltern rufen also ihn an, und lassen ihn die Heizung konfigurieren. 
Bild von Heizungsinterface wird gezeigt. Beschreibt die verschiedenen Tasten die zu 
sehen sind. 

\paragraph{Folie 5}
Wie lässt sich jetzt also der Sommer / Winterbetrieb einschalten? Sucht
Bedienungsanleitung - ist das erste Problem. Oft hat man sie nicht, optimal wäre
also wenn man das Gerät auch ohne Anleitung bedienen kann. 
Bild in Anleitung ist nicht sehr Aussagekräftig. Text ist wohl besser. 
Vorlauftemperatur verringert sich, nicht was er will. Heizung soll ja ausgehen.
Herr Kaindl ruft Wartungstechniker an, dieser ruft wieder einen Wartungstechniker an,
der sich speziell mit diesem Gerät auskennt. 
Temperatur musste einfach so weit verringert werden bis die Heizung von selbst
Sommerbetrieb anzeigt. 
Was war das Problem? Diskussion.
\begin{itemize}
\item Bedienungsanleitung nicht gut genug
\item Direkte Ersichtlichkeit nicht gegeben (Kostet aber vielleicht mehr, TradeOff
zwischen Usability und Funktionalität, Techniker kümmern sich nicht um Andere - 
muss für sie selbst funktionieren)
\end{itemize}

\paragraph{Folie 6}
Garage Entry Beispiel. 
Kleine Nebengeschichte: Beim Blachuta unter der Woche kann man das Ticket für die
Garage einlösen und bekommt eine Gutschrift. Gutschrift einlösen, 10Cent sind noch
zu zahlen! Warum? Betrag wurde erhöht, Gutschriftkarte wurde aber noch nicht aktualisiert.

Jetzt zum konkreten Beispiel.
Karte lässt sich nicht gut lesen beim Einstecken in das Gerät - kein Feedback vom
Gerät. Ärgerlich.

\paragraph{Folie 7}
Car Windows Beispiel. Kaindl mietet sich einen BWM für eine Tagung. Bekommt Schlüssel
und eine Ausfahrkarte. Alte Autos hatten Kurbeln für die Fensterheber. 
Kaindl muss also rausfahren und seine Karte stecken. Findet den Schalter für den
Fensterheber nicht, muss Türe öffnen um Karte zu stecken. 
Später stellt sich dann heraus: In der Mittelkonsole ist der Schalter für den 
Fensterheber. Keine gute Entdeckbarkeit. 

Automatik vs. Schaltung - etwas mit Automatik fahren, danach wieder mit Gangschaltung.
Man vergisst aus Gewohnheit auf das Kupplung treten. 

\paragraph{Folie 8}
Start Stop Mode vom Auto. Auto ausborgen, hat Start/Stop Automatik. Weiß es nicht,
wundert sich sehr. Warum ist der Motor aus?
Lösung: Auto hat ein Display, welches Start/Stop Modus auch anzeigt.
Wenn man allerdings jetzt telefonieren will, muss man erst die Meldung quittieren. 
Weiterentwicklung: Zeigt nur noch den Modus in einem kleinen Icon an, welches nicht
quittiert werden muss. 

\paragraph{Folie 9}
BWM Auto Schlüssel. 
Findet die Option für das Absperren des Autos am Schlüssel nicht. Stellt sich heraus
dass man das BMW Logo drücken muss um es zu schließen. 

Nebengeschichten:
Kaindl prueft immer ob die Tueren geschlossen sind. Eines Tages schlieszt das Auto
nicht - weisz nicht wie er sein Auto jetzt abschlieszen soll. 
Probiert in der Einsamkeit aus ob es an der Naehe des Schluessels liegt. Und so
war es auch. 

Bei einem anderen Auto konnte er zwar aktiv abschlieszen, aber wenn er die Fahrer
Tuer probiert hat dann geht es von selbst wieder auf. Die anderen Tueren bleiben
aber geschlossen. Muss man auch wissen.

\paragraph{Folie 11}
In-flight Infotainment. 
Fliegt gerade nach Liverpool, schlaeft ein bisschen im Flugzeug. Wacht auf, schaut
auf den Schirm, will dann etwas Anderes darauf machen. Findet aber den X-Knopf
nicht. Auch schlecht zu entdecken.
Anmerkung zu den Geschichten: jede Geschichte hat einen Kontext, deswegen erzaehlt
er es auch jedes Mal.

\paragraph{Folie 12}
Mehr Inflight Infotainment. Lustige Bedienung wird gezeigt - wo ist oben oder unten?
Man dreht das Teil um, und raeumt damit aus Versehen sein Essen / Getraenk ab. 
