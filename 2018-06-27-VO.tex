\section{2018-06-27 VO}
\paragraph{Human-Robot Interaction}

Was sind Roboter überhaupt? Was ist Human Robot Interaction? 

\begin{itemize}
\item Definition allgemein nicht möglich
\item Industrieroboter - Wiederprogrammierbare Maschine zur Ausführung von repetitiven
Tätigkeiten (Hier ist aber eher keine Interaktion erwünscht)
\item  Besitzt viele Aktoren - bei autonomen Robotern natürlich auch sehr viel Sensorik
\item Automated Guided Vehicles
\item Interessant sind Semi autonome Fahrzeuge
\item Humanoide - Nachbildung des Menschen (Welche Vorteile gegenüber zweckdienlichem
Roboter?)
\end{itemize}

Frage: Sind wir selbst überhaupt autonom? Auf uns alleine gestellt ist überleben
eigentlich auch schon sehr schwierig. Vollständige Autonomie ist meistens gar
nicht erwünscht.

Toxic Waste Management: Auch wieder Problematisch - Halbleiter die für Autonomie sorgen
sind von Strahlung genauso betroffen wie der Mensch.

Interessant sind Roboter für die Altenpflege, auch fraglich ob Nutzbarkeit in absehbarer
Zukunft.

\paragraph{What is special about robots?}


\begin{itemize}
\item Sensoren und Aktoren
\item Semi Autonom
\item Oft mobil einsetzbar
\item Verkörperung im Sinne einer physischen Hülle
	\subitem Impliziert Sicherheitsprobleme für Menschen
	\subitem Vertrauensgrundsatz kann nicht angenommen werden
\end{itemize}

\paragraph{What is special about Human Robot Interaction?}

\begin{itemize}
\item Interaktion im physischen Raum
\item Potentielle Sicherheitsgefahr
\item Interaktion mit 'intelligentem' Agenten
	\begin{itemize}
	\item{Unterschied zwischen Kommunikation und Interaktion? Interaktion mit Ziel der
	Kommunikation}
	\end{itemize}
\item Soziale Interaktion - der gesellschaftlichen Norm entsprechend verhalten
	\begin{itemize}
	\item{Beispiel Lift: Raumaufteilung im Lift gleichmäßig bei Menschen. Roboter
	steigt in Lift ein und stellt sich in den personal Space eines Menschen - sozial nicht
	akzeptiert}
	\end{itemize}
\item{Emotionen - können maximal emuliert werden. Kann der Roboter menschliche Emotionen
erkennen?}
\end{itemize}

\subsubsection{Einkaufswagen Roboter}

Motion Cue, auch wenn nicht Humanoid. Durch Introspektion inspiriert - wenn man etwas
zeigen möchte, dann dreht man sich auch zum Ziel und deutet hin. 
Low Level Gesten für den Einkaufswagen. 
Roboter auf dem Weg zum Shampoo - kommt man auf dem Weg an anderen Produkten vorbei die
auf der Einkaufsliste stehen, soll das Tempo verlangsamt werden um den Nutzer auf das
Produkt aufmerksam zu machen.
\textit{"Motion Cue durch Geschwindigkeitsveränderung"}

\subsubsection{Experiment mit dem Einkaufswagen}
Information über GUI und Sprache oder Information über Geschwindigkeit. 
Beim ersten Versuch waren die Leute etwas überrascht (First Time Usability), danach
war die Akzeptanz aber sehr gut.

\textbf{Within-Subject Design}: Nur bestimmte Anzahl an Versuchspersonen möglich,
beide Gruppen machen das Selbe, aber in unterschiedlicher Reihenfolge - um das Ergebnis
nicht zu Verfälschen. Lerneffekt kann dabei analysiert werden.

\subsubsection{Subjektive Ergebnisse des Experiments}
Darf man davon einen Mittelwert ausrechnen? Nein, bei subjektiven Werten (auch z.B Noten)
ist das Maß-theoretisch eigentlich nicht möglich. 
Median wäre im Prinzip okay.

\subsubsection{Discussion of Motion Cue}
Sozial nicht akzeptable fällt immer weniger auf - wenn der Roboter normal weiter fährt
fällt das auf. 
\subsubsection{Conclusion of Motion Cue}
Ohne Verkörperung ist das nicht möglich. 

\subsubsection{Robot Supported Cooperative Work (RSCW)}
Computer Supported Cooperative Work (CSCW) (Wie z.B bei E-Mail...).
Bei Robot darüber hinaus gehen und versuchen andere Facetten zu entdecken. 
Ohne Kommunikation keine Kooperation möglich - kooperatives Einkaufen, gemeinsame
Einkaufsliste (wird natürlich verteilt verwaltet). Beide kaufen gleichzeitig ein -
höhere Effizienz. 
Roboter kommuniziert direkt mit Einkaufswagen und Einkaufswagen kommuniziert direkt mit
anderem Einkaufswagen. DSL - Speech Act, Ontologie.

Shared Work Artifact - Shopping List
Awareness - Die eigene Aktivität im Kontext der anderen verstehen.

\subsubsection{Conclusion}

RSCW ist Erweiterung von CSCW.

Während der Bewegung ist der physische Kontext wichtig - wo ist der andere, kommt in CSCW
nicht vor dar statischer Ort ohne Interaktion mit Umwelt.
Emergent Behavior - Verhalten entsteht unvorhersehbar.

%\paragraph
%\\
%\noindent\fbox{%
%\parbox{\textwidth}{%
%\textbf{Storytime}:  
%}%
%}
%\\
