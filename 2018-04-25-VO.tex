\section{2018-04-25}
\paragraph{Wheel of Joy in Life}

Wichtiger Punkt, in der produktiven Welt aber fraglich - soll mir arbeiten mit 
SAP überhaupt Spaß machen?

\begin{itemize}
\item Connection: Ist den Leuten wichtig, siehe soziale Netzwerke.
\item Identity: Individuum  ist wichtig.
\item Accomplishment
\item Sensation
\end{itemize}

Definitionen auf Folie 10 sind sehr stark mit privaten Interessen verknüpft -
die Frage ist ob man hier einen professionelleren Fokus legen sollte.

Connection ist ein wichtiges Element - der Mensch ist eher ein Herdentier.
Identity - Celebration of Self. 
Sensation - \textit{Cool}, was ist das eigentlich?
Durch diese Zusammenfassung lässt sich User Experience noch eher erklären als 
wenn man sich nur auf den ISO-Standard bezieht.
Öffnung gegenüber der Tatsache, dass User Experience mehr ist als: \textit{Ich setze mich vor den Computer und erreiche etwas.}
\noindent\fbox{\parbox{\textwidth}{
\textbf{Storytime}: Goolge Glass - 90er, IEEE Konferenz zu Artificial Intelligence.
AI hat aktuell wieder einen Höhenflug, in den 90ern gerade am Abklingen.
Konferenzteilnehmer hat sich schon damals ein ähnliches Gerät gebaut - Prof. Kaindl
zweifelt Sinnhaftigkeit vorsichtig an. Erhält daraufhin die Diagnose Technophobia. }}
Warum hat sich Google Glass nicht durchgesetzt? Abschottung von Umwelt, Angst davor
gefilmt zu werden. 
Zum Vergleich GoPro: Wird in der Öffentlichkeit akzeptiert. 
Was macht die GoPro zum Erfolg? Sharing der Erfahrung, teilen auf sozialen Medien.
Dadurch gibt es dann \textbf{Connection} und \textbf{Identity}, wird dadurch 
insgesamt positiver wahrgenommen.
\paragraph{Challenged People}
Gesetzliche Lage: Jede Website sollte in der EU bereits Barrierefrei sein. Ist 
aber wohl eher nicht der Fall. Ist aber auch schwer zu erreichen.
Fokus auf visuelles Feedback natürlich problematisch, Ersatz durch Haptik?
\noindent\fbox{%
\parbox{\textwidth}{%
\textbf{Storytime}: Alter Wecker - eigentlich praktisch. Gutes haptisches Feedback
und statisches Interface (Knöpfe). 
}%
}
Behinderung durch älter werden passiert graduell: Daraus folgt, dass jeder der nicht
vorher stirbt irgendwann challenged ist. Und zweitens wird jeder früher oder später
in einer gewissen Weise challenged sein.

Man erkennt also, es sind Menschen die unsere Maschinen verwenden, das sollte immer
im Hinterkopf getragen werden.

\paragraph{User-Centered / Usage-Centered Design}

Needs (*9) sind eigentlich auch nur Requirements.

Für eine Klasse an Nutzern ein Modell erstellen - aber für mehrere Klassen sehr 
schwierig. Wie erkenne ich welche Klasse von Nutzern gerade meine Website nutzt?
Sieht man schon schlecht wird auch eine entsprechende Einstellung sehr schwer
gefunden werden.

Context Modeling: Gehe ich mit meinem Handy gerade in der Sonne oder im Haus - 
unterschiedliche Qualität der Nutzung.

\noindent\fbox{%
\parbox{\textwidth}{%
\textbf{Storytime}: Einrechen des Studienabschlusses. Auf Website einige Dinge 
anklicken, Zeugnisimport via TISS. 
Vor 10 Jahren gab es einen Übergang der bearbeitenden Sekräterinnen. Erneuerung
der Software direkt für neue Sekräterin - User Centered Design at its best!
Entwicklung direkt zusammen mit Sekräterin. 
}%
}

Daraus folgt, dass man einen Arbeitsprozess den man nicht kennt auch nicht
unterstützen kann.

\paragraph{Triangle of Joy in Use}

Direct into Action, Hassle Factor, The Delta. Jeder Schritt auf dem Weg zum Ziel
ist eine Erschwernis für den User.

Nutzung der Maus - Indirektion. Aktion außerhalb des PCs erzeugt Aktion am Bildschirm.
Touchscreen erzeugt direkte Interaktion mit User Interface.

Bankomaten - Tasten auf den Seiten des Bildschirms. Indirektion. 

\noindent\fbox{%
\parbox{\textwidth}{%
\textbf{Storytime}: Navi für Island ausgeborgt, hat keinen guten Touchscreen gehabt.
Eine Stelle war extrem oft in Verwendung - Ermüdung und Nichtfunktionieren des
Touchscreens an dieser Stelle.

Oder auch Automat in Stockholm für Autos. Ging ebenfalls an manchen Stellen gut
und an deren nicht.
}%
}

Daraus folgt: Konsistente Erfahrung besser, zur Not auch durch Indirektion.
