\section{2018-04-18 VO}
\paragraph{Definition of User Experience}

Diverse ISO Normen von den Folien.

Was ist mein Ziel wenn ich zum Bankomaten gehen?
Geld abheben ist kein Ziel - Mehr Bargeld haben (Man wird nicht reicher dadurch)
wollen ist ein Ziel. Ist aber eigentlich auch nicht wirklich das Ziel, eher ein
Teilziel. Ziel ist irgendetwas zu kaufen. Stichwort Auxilary Goals.

User Experience ISO Standards (*6) haben zu unserem Verständnis beigetragen.
Das Thema bleibt offen und wir kommen darauf zurück, unterschied zwischen
Usability und User Experience bleibt bis dahin eher schwammig.

\paragraph{Booking of Rental Car}
Wer hat schon einmal ein Mietauto gemietet? Erfahrungen? Metasuchanbieter vs.
Direktanbieter.
Versicherungen, wenn ja welche? Ohne Selbstbeteiligung - Herr Kaindl hat Unfall
auf iberischer Halbinsel. 
War alles inkludiert in der All Inclusive Versicherung? Glasbruch und Reifen
waren nicht in in Versicherung inkludiert. Für den Normaluser sind die AGB
nicht wirklich zu überschauen, durchzulesen. User hakerlt einfach ab.

Filter für versicherte Teile am Anfang der Autosuche und nicht wenn man erst 
das Auto ausgesucht hat. User Experience also insgesamt eher schlecht.

\paragraph{Human Factors}
User Experience besser verstehen. Menschen und User Interfaces. Resultate
aus der Psychologie anwenden. 
Challenged People - responsibilty, wenn man selbst kein Problem hat, dann fühlt
man sich auch nicht wirklich verantwortlich.

Schlechte User Interfaces sind oft ein Grund für das Nichtverwenden von Software.
SAP als Beispiel für schlechte User Interfaces - die können sich das Leisten,
da Marktdurchdringung sehr gut ist. Vizerektor hat 3 Kurse über SAP gehört, aber
musste es nie verwenden - fand es daher gut.

8 Stunden kein SAP weil Wartungsfenster im Bundesrechenzentrum, dort läuft es
nämlich.

Virtuelle Maschine mit Windows XP für SAP Client Software - Usability ist schlecht.
Usability und Security sind oft Gegner, kann man aber auch harmonieren lassen.

Beurteilung von Menschen nach Aussehen, schlecht wieder wegzubekommen. Gilt
auch für UIs.

Deshalb sollte man Human Factors beim Design miteinbeziehen.

\paragraph{Human Cognitive Abilities}
Menschen sind nicht gleich, haben unterschiedliche Fähigkeiten, aber auch 
Gemeinsamkeiten. Manche bevorzugen Grafiken, manche Text.

Menschen haben Langzeitgedächtnis und Kurzzeitgedächtnis. 
Geschichte über Tricks zum Merken, 7 Sachen oder so kann man sich merken,
die überlagern sich dann, man vergisst, manches bleibt. Mit Tricks macht man
Pointer im Kopf und kann sich so Dinge über Geschichten merken.

Tischtennnisspieler: Haben die bessere Reflexe als wir? Eher bessere
Antizipation beim Spiel, Erfahrung und Beobachtung. Reaktion darauf wie der Ball
fliegen wird.
Reaktionßeit hat ein Limit. 

Im Fehlerfall kommt es immer zu Folgefehlern - Stressreaktionen.

\paragraph{Cognitive Ergonomics}
Definition nicht auswendig lernen, aber wissen dass es das gibt und sich
Menschen damit beschäftigen. Verstehen, dass der Mensch gewisse Eigenschaften hat.

z.B. Wheel of Joy of Life. Nächste Woche!
