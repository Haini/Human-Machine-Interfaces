\section{2018-05-16}

\paragraph{Rendering of Final User Interfaces}

Optimierungen, Multimodale Interfaces (Speech Modell - Hin und Her Interaktion).
Bei sehr groszen Bildschirmen ist es auch moeglich mehrere Optionen gleichzeitig
auszuwaehlen. z.B Hin und Rueckflug Zeit sofort waehlen, anstatt zuerst Hin und dann
Rueckflug.

\paragraph{Usability Test}

Testen ist Fehler finden - Usability Testen ist Usability Fehler finden. Prototyp wird von
Menschen (Test subjects) ausprobiert. Werden dort dann beobachtet. Idealerweise dann
auch Videos. 
Warum ueberhaupt Videos aufzeichnen? Die laengste Zeit passiert nichts interessantes und
dann verpasst man die interessante Stelle. Auszerdem Annotation etc.

Testsubjekte sollten laut sprechen, sagen was sie Denken, wie sie gerade den Bildschirm
wahrnehmen. Problem der unterbewussten Wahrnehmung - oft schwierig auszudruecken was
gerade der echte Gedankengang ist.

Fokus sollte auf die Probleme und nicht auf die Loesungen gelegt werden.

Likert Scale: Sehr gut, gut, neutral, schlecht, sehr schlecht.

\paragraph{Usability Study}
Es sollen Vergleiche gemacht werden. z.B Tab Navigation vs Scroll Navigation, was ist
besser? Unterscheidung zwischen persoenlicher Praeferenz und echter Effizienz. Effizienz
kann Subjektiv verzerrt wahrgenommen werden.
Viele Studien mit vielen Studierende, vielleicht nicht ganz optimal.

Verschiedene Bildschirmgroeszen sorgen fuer verschiedene Ergebnisse. 

30 Studienteilnehmener - jeder bekommt beide User Interfaces. Reihenfolge sollte
unterschiedlich sein.

Vergleich zwischen Scrollen und Tabs. Statistische Signifikanz und Korrelation anschauen.
Adjusted Task Time: Zeit der relevanten Aktionen. Error Rate waehrend dem Test.

Null Hypothesen. Vertikales Interface war effizienter am Telefon. Am PC ist scrollen eher
schlecht.




%\paragraph
%\\
%\noindent\fbox{%
%\parbox{\textwidth}{%
%\textbf{Storytime}:  
%}%
%}
%\\

